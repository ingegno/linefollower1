\section{\engt{Appendix} \nedt{Appendix}}
\eng{TODO
}

\ned{Lijnvolg sketches
}

\subsection{\engt{Test Servo} \nedt{Test Servo's}}
\eng{todo}
\ned{Volgende code bevat hoe je de Servo motoren kunt controleren. De eerste sketch toont hoe je precies een Servo definieert in Arduino code, en hoe je opdrachten ernaar stuurt.}

\begin{code}[les01\_servo.ino]\label{c:a2_servo}
 \ \newline
\inputardfull{\string"../sketches/les01_servo/les01_servo.ino\string"}
\end{code}

\eng{todo}
\ned{Doe wat testen met bovenstaande code. Het is voor het verdere verloop belangrijk dat je functionaliteit van je robot opsplitst in deeltjes die apart niet moeilijk zijn om te verstaan. Zo hou je overzicht over het geheel. Onze volgende code herschrijft het aansturen van de servo motoren in zo'n codeblokjes, welke we functies noemen: je krijgt input, je doet daar iets mee, en dan eindig je.}

\begin{code}[les02\_servo\_functies]\label{c:a2_servo}
 \ \newline
\inputardfull{\string"../sketches/les02_servo_functies/les02_servo_functies.ino\string"}
\end{code}

\subsection{\engt{Test Distance Sensor} \nedt{Test Afstandssensor}}
\eng{todo}
\ned{Volgende code bevat een eenvoudige en een iets vluggere, maar moeilijkere, methode om de afstand tot een object te bepalen. Met een basiskennis Arduino kun je de code lezen. De workshop begeleider om raad vragen bij problemen!}

\begin{code}[les03\_afstandsensor.ino]\label{c:a1_distsens}
 \ \newline
\inputardfull{\string"../sketches/les03_afstandsensor/les03_afstandsensor.ino\string"}
\end{code}


%\begin{code}\label{c:l2_motor}
% \ \newline
%\inputard{\string"../sketches/calib_motors_rotate_axis/calib_motors_rotate_axis.ino\string"}{37}{54}
%\end{code}


